%%% \ONERQMTV[9] is macro for consistently formatted requirements
\ONERQMTVKPP
% #1 is requirement Number
{\RqtNumberBase.1}
% #2 is Title
{BDP-Heavy \DTV Waveform Transmission Input Processing}
% #3 is requirement label (expected to be of form rqt:XXX)
{rqt:Waveform_DTV_Heavy_Transmission_Input}
% #4 is Text of the specification
{The BDP-Heavy system shall receive the \MPEGTS data stream and output it as the Video Data Stream; all controlled by the parameters of the Waveform Control Inputs.}
% #5 is Status (S in {(T), (O), (I), (D)} listed by phase as \item [] S)
{
	\item [Phase 1]  Threshold
}
% #6 is Acceptance
{This requirement shall be verified by demonstration.}
% #7 is Traceability
{
	\item [5.1.1] Section in CDD for BDP FoS~\cite{ref__BDP_FOS_CDD}.
	\item [5.1.2] Section in CDD for BDP FoS~\cite{ref__BDP_FOS_CDD}.
	\item [5.5.1] Section in CDD for BDP FoS~\cite{ref__BDP_FOS_CDD}.
	\item [5.5.2] Section in CDD for BDP FoS~\cite{ref__BDP_FOS_CDD}.
	\item [5.5.3] Section in CDD for BDP FoS~\cite{ref__BDP_FOS_CDD}.
	\item [5.5.4] Section in CDD for BDP FoS~\cite{ref__BDP_FOS_CDD}.
}
% #8 is Notes; listed as enumeration \item ...
{
	\item This processing is enabled when the input selects the sub-mode, Digital Television (\DTV) in this case, with a supported DTV Video Standard.
}
% #9 is version of spec for changebars
{P2}
%%%%% end \ONERQMTV[9] macro

%%%%%%%%%%%%%%%%%%%%%%%%%%%%%%%%%%%%%%%%%%%%%%%%%%%%%%%%%%%%%%%%%%%%%%%%%%%%%%%%%%%%%%%%%%%%%%%%%%%%%%%%%%%%%%%%%%%%%%%%%%%%%%%%
%%%%%%%%%%%%%%%%%%%%%%%%%%%%%%%%%%%%%%%%%%%%%%%%%%%%%%%%%%%%%%%%%%%%%%%%%%%%%%%%%%%%%%%%%%%%%%%%%%%%%%%%%%%%%%%%%%%%%%%%%%%%%%%%

%%% \ONERQMTV[9] is macro for consistently formatted requirements
\ONERQMTVKPP
% #1 is requirement Number
{\RqtNumberBase.2}
% #2 is Title
{BDP-Heavy \DTV Waveform Transmission Output Processing}
% #3 is requirement label (expected to be of form rqt:XXX)
{rqt:Waveform_DTV_Heavy_Transmission_Output}
% #4 is Text of the specification
{The BDP-Heavy system shall randomize, encode, and interleave the data from the Video Data Stream; group the bits into symbols and modulate on the \RF carrier using the appropriate method dictated by the \DTV Video Standard; then convert to an analog signal creating the \RF Broadcast Pre-Amp Data Output.}
% #5 is Status (S in {(T), (O), (I), (D)} listed by phase as \item [] S)
{
	\item [Phase 1]  Threshold
}
% #6 is Acceptance
{This requirement shall be verified by demonstration.}
% #7 is Traceability
{
	\item [5.1.1] Section in CDD for BDP FoS~\cite{ref__BDP_FOS_CDD}.
	\item [5.1.2] Section in CDD for BDP FoS~\cite{ref__BDP_FOS_CDD}.
	\item [5.5.1] Section in CDD for BDP FoS~\cite{ref__BDP_FOS_CDD}.
	\item [5.5.2] Section in CDD for BDP FoS~\cite{ref__BDP_FOS_CDD}.
	\item [5.5.3] Section in CDD for BDP FoS~\cite{ref__BDP_FOS_CDD}.
	\item [5.5.4] Section in CDD for BDP FoS~\cite{ref__BDP_FOS_CDD}.
}
% #8 is Notes; listed as enumeration \item ...
{
	\item The \RF Broadcast Pre-Amp Data Output signal will have an unmodulated carrier output power level of −20 \dBm.
	\item The modulation method 8VSB or COFDM with PSK, QPSK, QAM, 16QAM, or 64QAM is controlled by the selected \DTV Video Standard.
	\item The average modulated \RF output power level will be estimated for use internally during the \DTV Waveform \VSWR Processing.
}
% #9 is version of spec for changebars
{P2}
%%%%% end \ONERQMTV[9] macro

