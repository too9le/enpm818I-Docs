%%% \ONERQMTV[9] is macro for consistently formatted requirements
\ONERQMTVKPP
% #1 is requirement Number
{\RqtNumberBase.1}
% #2 is Title
{BDP-Light \FM Waveform \VSWR Input Processing}
% #3 is requirement label (expected to be of form rqt:XXX)
{rqt:Waveform_FM_Light_VSWR_Input}
% #4 is Text of the specification
{The BDP-Light system shall calculate the Average Reflected Power from the \FM Received \RF Data signal, all controlled by the parameters of the Waveform Control Inputs.}
% #5 is Status (S in {(T), (O), (I), (D)} listed by phase as \item [] S)
{
	\item [Phase 1] Threshold
}
% #6 is Acceptance
{This requirement shall be verified by demonstration.}
% #7 is Traceability
{
	\item [5.1.1] Section in CDD for BDP FoS~\cite{ref__BDP_FOS_CDD}.
	\item [5.1.2] Section in CDD for BDP FoS~\cite{ref__BDP_FOS_CDD}.
	\item [5.5.1] Section in CDD for BDP FoS~\cite{ref__BDP_FOS_CDD}.
	\item [5.5.3] Section in CDD for BDP FoS~\cite{ref__BDP_FOS_CDD}.
	\item [5.5.4] Section in CDD for BDP FoS~\cite{ref__BDP_FOS_CDD}.
	\item [5.5.21] Section in CDD for BDP FoS~\cite{ref__BDP_FOS_CDD}.
}
% #8 is Notes; listed as enumeration \item ...
{
	\item This processing is enabled when the input selects the sub-mode, \FM or \FM with \RDS.
	\item The inputs will supply the \RF carrier frequency and frequency deviation information needed to extract the reflected power from the Received \RF Data Input.
	\item A digitally sampled signal that represents the reflected \RF signal is generated which is controlled by the carrier frequency and modulation bandwidth inputs.
}
% #9 is version of spec for changebars
{P2}
%%%%% end \ONERQMTV[9] macro

%%%%%%%%%%%%%%%%%%%%%%%%%%%%%%%%%%%%%%%%%%%%%%%%%%%%%%%%%%%%%%%%%%%%%%%%%%%%%%%%%%%%%%%%%%%%%%%%%%%%%%%%%%%%%%%%%%%%%%%%%%%%%%%%
%%%%%%%%%%%%%%%%%%%%%%%%%%%%%%%%%%%%%%%%%%%%%%%%%%%%%%%%%%%%%%%%%%%%%%%%%%%%%%%%%%%%%%%%%%%%%%%%%%%%%%%%%%%%%%%%%%%%%%%%%%%%%%%%

%%% \ONERQMTV[9] is macro for consistently formatted requirements
\ONERQMTVKPP
% #1 is requirement Number
{\RqtNumberBase.2}
% #2 is Title
{BDP-Light \FM Waveform \VSWR Output Processing}
% #3 is requirement label (expected to be of form rqt:XXX)
{rqt:Waveform_FM_Light_VSWR_Output}
% #4 is Text of the specification
{The BDP-Light system shall integrate the estimated forward-transmitted power and reflected-power measurements and then calculate the \VSWR value, generating fault notifications for erroneous conditions.}
% #5 is Status (S in {(T), (O), (I), (D)} listed by phase as \item [] S)
{
	\item [Phase 1] Threshold
}
% #6 is Acceptance
{This requirement shall be verified by demonstration.}
% #7 is Traceability
{
	\item [5.1.1] Section in CDD for BDP FoS~\cite{ref__BDP_FOS_CDD}.
	\item [5.1.2] Section in CDD for BDP FoS~\cite{ref__BDP_FOS_CDD}.
	\item [5.5.1] Section in CDD for BDP FoS~\cite{ref__BDP_FOS_CDD}.
	\item [5.5.3] Section in CDD for BDP FoS~\cite{ref__BDP_FOS_CDD}.
	\item [5.5.4] Section in CDD for BDP FoS~\cite{ref__BDP_FOS_CDD}.
	\item [5.5.21] Section in CDD for BDP FoS~\cite{ref__BDP_FOS_CDD}.
}
% #8 is Notes; listed as enumeration \item ...
{
	\item The integration time from the control inputs determines number of samples used in the calculations and the update rate.
	\item Low reflected power or high \VSWR measurements will generate the fault notifications which will be used to disable the \RF Broadcast Output. A low reflected power measurement while transmitting would indicate that no \RF power is being transmitted.
}
% #9 is version of spec for changebars
{P2}
%%%%% end \ONERQMTV[9] macro