%%% SVN stuff
\svnidlong
{$HeadURL: https://svn.riouxsvn.com/kneadlatxinputs/ExampleArtifactFolders/0%20-%20OCD/OCD_Chapter_08.tex $}
{$LastChangedDate: 2024-01-03 22:07:16 -0500 (Wed, 03 Jan 2024) $}
{$LastChangedRevision: 51 $}
{$LastChangedBy: KneadProject $}
\svnid{$Id: OCD_Chapter_08.tex 51 2024-01-04 03:07:16Z KneadProject $}

\chapter{Analysis of the proposed system}
\label{chap:Analysis of the proposed system}
\DIDINFO{OCD-8.0 :: This chapter shall be divided into the following sections to describe the analysis of the new system or the expected modified system.
This chapter can be considered to be an executive summary of the new/proposed systems.
The contents somewhat follow the common NABC (need, approach, benefit, competition) way of presenting a short summary of an idea.
The need, approach, and benefit are rolled up into the first section, while the competition is distributed in the final two sections.
}



This chapter is \TBD.

\section{Summary of advantages}
\label{sec:Summary of advantages}
\DIDINFO{OCD-8.1 :: This paragraph shall provide a qualitative and quantitative summary of the advantages to be obtained from the new or modified system. 
This summary shall include new capabilities, enhanced capabilities, and improved performance, as applicable, and their relationship to deficiencies identified in 4.1.}

This systems major advantages remove the human error when it comes to espresso. The monitoring of the water level potentially
protects the system from a user who forgets to check the water levels. It it quicker and more accurate for the embedded system
to manage the pumps and output products of the espresso machine. With the major advantage of more accurate espresso shots
as well as a more accurate summary of the shot. This platform also provides user with avenues to monitor and review previous shots.
A feature that does not currently exist on the market.

\section{Summary of disadvantages/limitations}
\label{sec:Summary of disadvantages/limitations}
\DIDINFO{OCD-8.2 :: This paragraph shall provide a qualitative and quantitative summary of disadvantages or limitations of the new or modified system. 
These disadvantages and limitations shall include, as applicable, degraded or missing capabilities, degraded or less-than-desired performance, greater-than-desired use of computer hardware resources, undesirable operational impacts, conflicts with user assumptions, and other constraints.
}

Some disadvantages of this system. There is still a user input. The user must weight and input the coffee they are going to use.
The system is also dependant on user competence. Understanding how to manipulate and change variables to get the desired effect.

The system additionally creates added complexity and more areas for failures. This system integrates into the existing machine 
and the user must install and learn how to operate it. 

\section{Alternatives and trade-offs considered}
\label{sec:Alternatives and trade-offs considered}
\DIDINFO{OCD-8.3 :: This paragraph shall identify and describe major alternatives considered to the system or its characteristics, the trade-offs among them, and rationale for the decisions reached.}

The biggest alternative to the \ThisSystem is a traditional espresso setup. A simple scale and timer with a user
controlling the system. This is a cost effective solution. It leaves rooms for human error but is how many users
brew espresso.

One additional trade off is the complexity of the system. The user must be comfortable installing the \ThisSystem.
This requires opening up the machine working with hardware meant to carry 120V AC current.
