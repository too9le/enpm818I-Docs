%%% SVN stuff
\svnidlong
{$HeadURL: https://svn.riouxsvn.com/kneadlatxinputs/ExampleArtifactFolders/0%20-%20OCD/OCD_Chapter_08.tex $}
{$LastChangedDate: 2024-01-03 22:07:16 -0500 (Wed, 03 Jan 2024) $}
{$LastChangedRevision: 51 $}
{$LastChangedBy: KneadProject $}
\svnid{$Id: OCD_Chapter_08.tex 51 2024-01-04 03:07:16Z KneadProject $}

\chapter{Analysis of the proposed system}
\label{chap:Analysis of the proposed system}
\input{DIDINFO_Snippets/OCD/OCD_8.0_DIDINFO.tex}

This chapter is \TBD.

\section{Summary of advantages}
\label{sec:Summary of advantages}
\input{DIDINFO_Snippets/OCD/OCD_8.1_DIDINFO.tex}

This systems major advantages remove the human error when it comes to espresso. The monitoring of the water level potentially
protects the system from a user who forgets to check the water levels. It it quicker and more accurate for the embedded system
to manage the pumps and output products of the espresso machine. With the major advantage of more accurate espresso shots
as well as a more accurate summary of the shot. This platform also provides user with avenues to monitor and review previous shots.
A feature that does not currently exist on the market.

\section{Summary of disadvantages/limitations}
\label{sec:Summary of disadvantages/limitations}
\input{DIDINFO_Snippets/OCD/OCD_8.2_DIDINFO.tex}

Some disadvantages of this system. There is still a user input. The user must weight and input the coffee they are going to use.
The system is also dependant on user competence. Understanding how to manipulate and change variables to get the desired effect.

The system additionally creates added complexity and more areas for failures. This system integrates into the existing machine 
and the user must install and learn how to operate it. 

\section{Alternatives and trade-offs considered}
\label{sec:Alternatives and trade-offs considered}
\input{DIDINFO_Snippets/OCD/OCD_8.3_DIDINFO.tex}

The biggest alternative to the \ThisSystem is a traditional espresso setup. A simple scale and timer with a user
controlling the system. This is a cost effective solution. It leaves rooms for human error but is how many users
brew espresso.

One additional trade off is the complexity of the system. The user must be comfortable installing the \ThisSystem.
This requires opening up the machine working with hardware meant to carry 120V AC current.
