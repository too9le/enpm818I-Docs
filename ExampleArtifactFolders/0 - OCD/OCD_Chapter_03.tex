%%% SVN stuff
\svnidlong
{$HeadURL: https://svn.riouxsvn.com/kneadlatxinputs/ExampleArtifactFolders/0%20-%20OCD/OCD_Chapter_03.tex $}
{$LastChangedDate: 2024-01-03 22:07:16 -0500 (Wed, 03 Jan 2024) $}
{$LastChangedRevision: 51 $}
{$LastChangedBy: KneadProject $}
\svnid{$Id: OCD_Chapter_03.tex 51 2024-01-04 03:07:16Z KneadProject $}


\chapter{Current system or situation}
\label{chap:Current System or Situation}
\input{DIDINFO_Snippets/OCD/OCD_3.0_DIDINFO.tex}

This chapter describes the capabilities and needs for the \ThisSystem. This \ThisSystem aims to solve the complicated world of specialty coffee.
New users may find themselves overwhelmed by the complexity of Espresso. The \ThisSystem focuses on usability to allow the user to input factors
like coffee weight and a desired brew ratio and allow the machine to automate the process. This system also monitors the Rancillio Silvia's 
current state, measuring water level to protect the system from running out of water and potentially damaging the boiler. At this time the system
is designed as a prototype to prove out capabilities. The focus of this design is time to market with the most user features.  


\section{Background, objectives, and scope}
\label{sec:Background, Objectives, and Scope}
\input{DIDINFO_Snippets/OCD/OCD_3.1_DIDINFO.tex}

This system is designed to meet the criteria of ENPM818I. This course focuses on embedded software design and servers to document the process of
developing an embedded system. 

The \ThisSystem aims to simplify the complicated process of brewing espresso for optimal user enjoyment. This system integrates with the existing
Rancillio Silvia espresso machine and must be safe and easy for users.


\section{Operational policies and constraints}
\label{sec:Operational Policies and Constraints}
\input{DIDINFO_Snippets/OCD/OCD_3.2_DIDINFO.tex}

This section is \TBD.


\section{Description of current system or situation}
\label{sec:Description of current system or situation}
\input{DIDINFO_Snippets/OCD/OCD_3.3_DIDINFO.tex}

This system is to operate in one mode. It is assumed that the system is operating at room temperature. The system will be interacted with
using a rotery encoder and a LCD display to provide user feedback. This system is expected to work with user guidance and it is expected
that the user will supervise the system.

The rest is \TBD

\section{Users or involved personnel}
\label{sec:Users or involved personnel}
\input{DIDINFO_Snippets/OCD/OCD_3.4_DIDINFO.tex}

The \ThisSystem is designed for two main users. Experienced coffee enthusiasts and new coffee users. The core 
feature of the \ThisSystem is the ability to set a brew ratio and guarantee the correct amount of output liquid. 
In addition the system provides water level monitoring and tracking data for each shot of espresso, a key feature
regardless of your skill level.

An experienced coffee enthusiasts gains the convince of automation in their process and the accuracy an automated system
provides allowing for more consistent espresso shots. In addition the monitoring of shot data allows additional Information
for the user to tweak in order to perfect their espresso.

A new user may not have the know how to pull their own shots. The automation provides a simple and effective interface
to get the user pulling the right ratio of coffee to water. In addition the \ThisSystem provides basic feedback to 
let the user know if they may have an under or over extracted espresso. This system aims to reduce variables to allow for
repeatable and consistent espresso without months or years of training.

\section{Support concept}
\label{sec:Support concept}
\input{DIDINFO_Snippets/OCD/OCD_3.5_DIDINFO.tex}

This section is \TBD.
