%%% SVN stuff
\svnidlong
{$HeadURL: https://svn.riouxsvn.com/kneadlatxinputs/ExampleArtifactFolders/0%20-%20OCD/OCD_Chapter_03.tex $}
{$LastChangedDate: 2024-01-03 22:07:16 -0500 (Wed, 03 Jan 2024) $}
{$LastChangedRevision: 51 $}
{$LastChangedBy: KneadProject $}
\svnid{$Id: OCD_Chapter_03.tex 51 2024-01-04 03:07:16Z KneadProject $}


\chapter{Current system or situation}
\label{chap:Current System or Situation}
\DIDINFO{OCD-3.0 :: This section shall be divided into the following paragraphs to describe the system or situation as it currently exists.
For existing systems, this chapter provides a summary of the performance (\SPS) and/or segment capabilities (\SSS) attributes; section numbers are shown in parenthesis in following DIDINFO blocks.
For new systems, this chapter provides a summary of the problem that needs to be addressed by the new system.}



This chapter describes the capabilities and needs for the \ThisSystem. This \ThisSystem aims to solve the complicated world of specialty coffee.
New users may find themselves overwhelmed by the complexity of Espresso. The \ThisSystem focuses on usability to allow the user to input factors
like coffee weight and a desired brew ratio and allow the machine to automate the process. This system also monitors the Rancillio Silvia's 
current state, measuring water level to protect the system from running out of water and potentially damaging the boiler. At this time the system
is designed as a prototype to prove out capabilities. The focus of this design is time to market with the most user features.  


\section{Background, objectives, and scope}
\label{sec:Background, Objectives, and Scope}
\DIDINFO{OCD-3.1 :: This paragraph shall describe the background, mission or objectives, and scope of the current system or situation.
Note that this section basically summarizes the normal chapter 1 boilerplate material and system overview from existing documentation.
}

This system is designed to meet the criteria of ENPM818I. This course focuses on embedded software design and servers to document the process of
developing an embedded system. 

The \ThisSystem aims to simplify the complicated process of brewing espresso for optimal user enjoyment. This system integrates with the existing
Rancillio Silvia espresso machine and must be safe and easy for users.


\section{Operational policies and constraints}
\label{sec:Operational Policies and Constraints}
\DIDINFO{OCD-3.2 :: This paragraph shall describe any operational policies and constraints that apply to the current system or situation.
}

This section is \TBD.


\section{Description of current system or situation}
\label{sec:Description of current system or situation}
\DIDINFO{OCD-3.3 :: This paragraph shall provide a description of the current system or situation.
Note that this is basically a summary of the detailed of \SPS and/or \SSS items. 
The description starts by identifying differences associated with different states or modes of operation (for example, regular, maintenance, training, degraded, emergency, alternative-site, wartime, peacetime). 
The distinction between states and modes is arbitrary. 
A system may be described in terms of states only, modes only, states within modes, modes within states, or any other scheme that is useful. 
If the system operates without states or modes, this paragraph shall so state, without the need to create artificial distinctions. 
The description shall include, as applicable:
\begin{itemize}[itemindent=5pt,topsep=0pt,itemsep=0pt,partopsep=0pt, parsep=0pt]
\item The operational environment and its characteristics (3.2),
\item Interfaces to external systems or procedures (3.2)
\item Charts and accompanying descriptions depicting inputs, outputs, data flow, and manual
and automated processes sufficient to understand the current system or situation from the
user’s point of view (3.2),
\item Capabilities/functions of the current system (3.3),
\item Performance characteristics, such as speed, throughput, volume, and frequency (3.3),
\item Major system components and the interconnections among these components (3.4 and 3.5),
\item Quality attributes, such as reliability, maintainability, availability, flexibility, portability,
usability, or efficiency, (3.11) and
\item Provisions for safety, security, privacy, (3.7, 3.8) and continuity of operations in emergencies (3.11).
\end{itemize}
}

This system is to operate in one mode. It is assumed that the system is operating at room temperature. The system will be interacted with
using a rotery encoder and a LCD display to provide user feedback. This system is expected to work with user guidance and it is expected
that the user will supervise the system.

The rest is \TBD

\section{Users or involved personnel}
\label{sec:Users or involved personnel}
\DIDINFO{OCD-3.4 :: This paragraph shall describe the types of users of the system, or personnel involved in the current situation, including, as applicable, organizational structures, training/skills, responsibilities, activities, and interactions with one another.
Note that this section is a summary of items found in the security and privacy (3.8), personnel (3.13), and training (3.14) sections of an \SPS or \SSS.}

The \ThisSystem is designed for two main users. Experienced coffee enthusiasts and new coffee users. The core 
feature of the \ThisSystem is the ability to set a brew ratio and guarantee the correct amount of output liquid. 
In addition the system provides water level monitoring and tracking data for each shot of espresso, a key feature
regardless of your skill level.

An experienced coffee enthusiasts gains the convince of automation in their process and the accuracy an automated system
provides allowing for more consistent espresso shots. In addition the monitoring of shot data allows additional Information
for the user to tweak in order to perfect their espresso.

A new user may not have the know how to pull their own shots. The automation provides a simple and effective interface
to get the user pulling the right ratio of coffee to water. In addition the \ThisSystem provides basic feedback to 
let the user know if they may have an under or over extracted espresso. This system aims to reduce variables to allow for
repeatable and consistent espresso without months or years of training.

\section{Support concept}
\label{sec:Support concept}
\DIDINFO{OCD-3.5 :: This paragraph shall provide an overview of the support concept for the current system, including, as applicable to this document, support agency(ies); facilities; equipment; support software; repair/replacement criteria; maintenance levels and cycles; and storage, distribution, and supply methods.
Note that this is a summary of items found in the \SPS or \SSS logistics (3.15), other (3.16), and packaging (3.17) sections.}

This section is \TBD.
