An overview the \FM \RF receive processing performed is shown in Figure~\ref{fig:Light_RF_Processing_FM_Receive}.

\begin{figure}[htbp]
	\centering
		\includegraphics[width=6.5in]{../zProjectWideData/images/RFProcessing_Light_FM_Receive_300dpi_6500millWide.eps}
	\caption[BDP-Light FM RF Receive Processing Overview]{BDP-Light FM RF Receive Processing Overview}
	\label{fig:Light_RF_Processing_FM_Receive}
\end{figure}

%%% \ONERQMTV[9] is macro for consistently formatted requirements
\ONERQMTV
% #1 is requirement Number
{\RqtNumberBase.1}
% #2 is Title
{\ThisSubSegment Receive Processing}
% #3 is requirement label (expected to be of form rqt:XXX)
{rqt:Light_RF_FM_Receive_Processing}
% #4 is Text of the specification
{
The BDP-Light system shall receive externally radiated \RF signals, filter for the commercial \FM broadcasting band, and amplify. The BDP-Light FM Receive Processing capability shall have the following properties:
\begin{my_enumerate}
	\item The Band-Pass Filter (\BPF) will filter out signals outside of the commercial \FM broadcasting band of 87 - 108 \MHz.
	\item The Low-Noise Amplifier (\LNA) will amplify the \FM bandwidth input signal(s) by greater than 40\dB with a Noise Figure less than 2 \dB and a maximum output power less than 0 \dBm.
	\item The Antenna will receive the radiated \RF signals in the \FM band and should be a directional antenna for maximum performance. The antenna \HAAT is dependent on the location of the transmitter, but higher is better to reduce signal fading.
\end{my_enumerate}
}
% #5 is Status (S in {(T), (O), (I), (D)} listed by phase as \item [] S)
{
	\item [Phase 1] Objective
}
% #6 is Acceptance
{This requirement shall be verified by demonstration.}
% #7 is Traceability
{
	\item [5.4] Table in CDD for BDP FoS~\cite{ref__BDP_FOS_CDD}.
	\item [5.5.5] Section in CDD for BDP FoS~\cite{ref__BDP_FOS_CDD}.
}
% #8 is Notes; listed as enumeration \item ...
{
\item N/A
	%\item The Band-Pass Filter (\BPF) will filter out signals outside of the commercial \FM broadcasting band of 87 – 108 \MHz.
	%\item The Low-Noise Amplifier (\LNA) will amplify the \FM bandwidth input signal(s) by greater than 40\dB with a Noise Figure less than 2 \dB and a maximum output power less than 0 \dBm.
	%\item The Antenna will receive the radiated \RF signals in the \FM band and should be a directional antenna for maximum performance. The antenna \HAAT is dependent on the location of the transmitter, but higher is better to reduce signal fading.
}
% #9 is version of spec for changebars
{P2}
%%%%% end \ONERQMTV[9] macro

