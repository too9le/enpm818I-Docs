
%%% \ONERQMTV[9] is macro for consistently formatted requirements
\ONERQMTV
% #1 is requirement Number
{\RqtNumberBase.1}
% #2 is Title
{\ThisSubSegment Transmission Processing}
% #3 is requirement label (expected to be of form rqt:XXX)
{rqt:Light_RF_TVVHFLo_Transmission_Processing}
% #4 is Text of the specification
{
The BDP-Light system shall filter, amplify, and radiate the \TVVHFLow \RF Broadcast Pre-Amp Data along with producing an \RF signal that is representative of the reflected \RF power. The BDP-Light VHF-L RF Transmission Processing capability shall have the following properties:
\begin{my_enumerate}
	\item The TV \RF Broadcast Pre-Amp Data signal will have an average output power level of -15 \dBm per \RF channel.
	\item The Band-Pass Filters (\BPF) will filter out signals outside of the commercial \FM broadcasting band of 40 - 100 \MHz.
	\item The \RF Power Amplifier (\PA) will amplify the \TVVHFLow bandwidth input signal(s) by 60\dB and have at least 25 W maximum output power. These power levels will support a single TV broadcast (TV station) at the maximum 8-mile range.
	\item The Low-Pass Filter (\LPF) will filter out the generated harmonics above 100 \MHz.
	\item The Directional Coupler (\DC) will pass the \RF signal to the antenna while siphoning off a portion (-30 to -50 \dB) of the reflected power.
	\item The Antenna will radiate the \RF signal(s) in the \TVVHFLow band and satisfy the range requirement using an omnidirectional dipole antenna. The antenna will have a \VSWR less than 1.8 over the \TVVHFLow band, be capable of handling the \PA output power, and need to be mounted with a Height Above Average Terrain (\HAAT) of at least 35 feet.
	\item The attenuator is needed to protect the receiver from being overdriven by the reflected power. The value is dependent on the maximum \PA output power and coupling factor of the \DC. The attenuator will limit the signal to 0 \dBm.
\end{my_enumerate}
}
% #5 is Status (S in {(T), (O), (I), (D)} listed by phase as \item [] S)
{
	\item [Phase 1]  Objective
}
% #6 is Acceptance
{This requirement shall be verified by demonstration.}
% #7 is Traceability
{
	\item [5.1.1] Section in CDD for BDP FoS~\cite{ref__BDP_FOS_CDD}.
	\item [5.1.2] Section in CDD for BDP FoS~\cite{ref__BDP_FOS_CDD}.
	\item [5.1.4] Section in CDD for BDP FoS~\cite{ref__BDP_FOS_CDD}.
	\item [5.5.1] Section in CDD for BDP FoS~\cite{ref__BDP_FOS_CDD}.
	\item [5.5.3] Section in CDD for BDP FoS~\cite{ref__BDP_FOS_CDD}.
	\item [5.5.4] Section in CDD for BDP FoS~\cite{ref__BDP_FOS_CDD}.
}
% #8 is Notes; listed as enumeration \item ...
{
\item N/A
	%\item The TV \RF Broadcast Pre-Amp Data signal will have an average output power level of −15 \dBm per \RF channel.
	%\item The Band-Pass Filters (\BPF) will filter out signals outside of the commercial \FM broadcasting band of 40 – 100 \MHz.
	%\item The \RF Power Amplifier (\PA) will amplify the \TVVHFLow bandwidth input signal(s) by 60\dB and have at least 25 W maximum output power. These power levels will support a single TV broadcast (TV station) at the maximum 8-mile range.
	%\item The Low-Pass Filter (\LPF) will filter out the generated harmonics above 100 \MHz.
	%\item The Directional Coupler (\DC) will pass the \RF signal to the antenna while siphoning off a portion (-30 to -50 \dB) of the reflected power.
	%\item The Antenna will radiate the \RF signal(s) in the \VHF-Low band and satisfy the range requirement using an omnidirectional dipole antenna. The antenna will have a \VSWR less than 1.8 over the \VHF-Low band, be capable of handling the \PA output power, and need to be mounted with a Height Above Average Terrain (\HAAT) of at least 35 feet.
	%\item The attenuator is needed to protect the receiver from being overdriven by the reflected power. The value is dependent on the maximum \PA output power and coupling factor of the \DC. The attenuator will limit the signal to 0 \dBm.
}
% #9 is version of spec for changebars
{P2}
%%%%% end \ONERQMTV[9] macro

