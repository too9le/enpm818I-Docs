\DIDINFO{This paragraph shall summarize the purpose and contents of this document and shall describe any security or privacy considerations associated with its use.}

This document format is based upon the guidance in the \STR{} \DID~\cite{ref__STR_DID}.
The test results are documented following the guidelines of ISO-12207~\cite{ref__ISO_12207} and MIL-STD-498~\cite{ref__MIL_STD_498} (from which ISO-12207 originated).
This document follows the listed \STD sub-section order.
\begin{description}
	\item[Section 1] provides an overview of the system and this document.
	\item[Section 2] lists general and application-specific reference documents as well as glossary terms and acronyms. 
	\item[Section 3] summarizes the test results.
	\item[Section 4] details the tests results. 
	\item[Section 5] documents, as applicable, the test parameters beyond those recorded in the test itself and provides a log of the detailed test and test case description and results.
	\item[Section 6] if needed, lists any general notes as may be applicable.
	\item[Appendices] if needed, provide additional information regarding the test results.
\end{description}


This document also is structured to allow the \STD / \STS artifact to serve as the basis for the system test report (\STR).
Each test is supplied with spaces for capturing pertinent hardware, software, and other log information.
Each test is divided into one or more test cases, each with detailed steps, expected results for each step, and a verification mark for each test step.
All tests steps also provide space to fill in the results and to write notes and comments about each test step.
The goal of this style is to generate this \STR by scanning in the resultant \STD / \STS artifact with comments to provide a log of the test results.
In this manner, a ``written'' record of the testing is generated, thus saving money by not requiring a completely separate recording document for the \STR.