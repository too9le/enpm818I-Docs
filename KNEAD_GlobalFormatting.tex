%%%%%%%%%%%%%%%%%%%%%%%%%%%%%%%%%%%%%%%%%%%%%%%%%%%%%%%%%%%%%%%%%%%%%%%%%%%%%%%%%%%%%%%%%%%%%%%%%%%%%%%%%%%%%%%%%%%%%
% BEGIN :: KNEAD \usepackage section
%          MUST BE FIRST
%          list all globally needed packages
%%%%%%%%%%%%%%%%%%%%%%%%%%%%%%%%%%%%%%%%%%%%%%%%%%%%%%%%%%%%%%%%%%%%%%%%%%%%%%%%%%%%%%%%%%%%%%%%%%%%%%%%%%%%%%%%%%%%%
%

%from MILSTD498.sty

%
% Circled Number and Letter Icons
%
\usepackage{tikz}
\newcommand*\circled[1]{\tikz[baseline=(char.base)]{
    \node[shape=circle, draw, inner sep=1pt, 
        minimum height=12pt] (char) {\vphantom{1g}#1};}}

% at top of file
\usepackage{svn-multi}

%at bottom of file
\usepackage{indentfirst}%WLC20210315 -- added to get 1st lines in sections and below to be indented.
%% This is needed to prevent odd reading when sentences occur after lists and other formatting issues.


%%% from MILSTD498document.tex
\usepackage[shortlabels]{enumitem}%allows [] options for spacing, [shortlabels] allows ``a)'' as in \begin{enumerate}[a),itemsep=-1ex,topsep=0pt]
\usepackage{etoolbox}% added 20210913 for the toggle capability
\usepackage[nodayofweek]{datetime}
\newdateformat{mydate}{\twodigit{\THEDAY}{ }\shortmonthname[\THEMONTH], \THEYEAR}
\mydate

\usepackage{titlesec}

%%%
%%% Set default of showing all 5 levels in TOC
%%% %part,ch,sec,ssec,sssec,para,subpara
%%%   -1,  0,  1,  2,    3,   4,    5
\setcounter{secnumdepth}{5} 
\setcounter{tocdepth}{5}    

\usepackage{csvsimple,longtable,booktabs}


\usepackage{ifthen}
\usepackage{makeidx}
%\usepackage{showidx}
\usepackage{amsmath}
\usepackage{gensymb}
\usepackage{float}
\usepackage{subfig}
\usepackage{dtklogos}
% \usepackage{setspace}
\usepackage[normalem]{ulem}

\usepackage{pifont}% http://ctan.org/pkg/pifont
\newcommand{\cmark}{\ding{51}}%
\newcommand{\xmark}{\ding{55}}%

%%% 20141212WLC - had issues so I edited the text instead
\usepackage{longtable}
\setcounter{LTchunksize}{100}

\usepackage{graphicx}
% \usepackage[space]{grffile} %to get file names with spaces to work
%https://tex.stackexchange.com/questions/8422/how-to-include-graphics-with-spaces-in-their-path

\usepackage{afterpage}
\usepackage{booktabs}

\usepackage{color}
\definecolor{MILSTD498BLUE}{RGB}{0, 96, 192}
\definecolor{GREEN}{RGB}{0, 192, 0}
\definecolor{YELLOW}{RGB}{192, 192, 0}
\definecolor{RED}{RGB}{255, 0, 0}
\definecolor{BLUE}{RGB}{0, 0, 192}

\usepackage{listings}
\lstset{ %
lineskip=-2pt,
xleftmargin=.2in,               % slide left margin to the right to keep line numbers inside margin
language=C++,                   % choose the language of the code
alsolanguage=C,                 % choose an alternative language of the code
basicstyle=\footnotesize,       % the size of the fonts that are used for the code
numbers=left,                   % where to put the line-numbers
numberstyle=\footnotesize,      % the size of the fonts that are used for the line-numbers
stepnumber=1,                   % the step between two line-numbers. If it is 1 each line will be numbered
numbersep=5pt,                  % how far the line-numbers are from the code
backgroundcolor=\color{white},  % choose the background color. You must add \usepackage{color}
showspaces=false,               % show spaces adding particular underscores
showstringspaces=false,         % underline spaces within strings
showtabs=false,                 % show tabs within strings adding particular underscores
frame=single,                   % adds a frame around the code
tabsize=2,                      % sets default tabsize to 2 spaces
captionpos=b,                   % sets the caption-position to bottom
breaklines=true,                % sets automatic line breaking
breakatwhitespace=false,        % sets if automatic breaks should only happen at whitespace
escapeinside={\%*}{*)}          % if you want to add a comment within your code
}

%%% tell LaTeX to allow for line breaks at underscores since we have lots oflong names with them
\renewcommand\_{\textunderscore\allowbreak}

\usepackage{xspace}%so we don't have to put braces at end of commands
%%% make sure font changing commands have double braces to encapsulate the font area
%%% e.g. see TBD macro below

%%%
%%% commands to mark text so it can be included showing notes or other such things but be marked as being "marked" as special.
%%% following command provides a common way to show marked text is "special".

\usepackage{framed}
\newcommand{\MarkThisText}[1]{\begin{framed} #1 \end{framed}}
\newcommand{\MarkedTextStatement}[1]{\MarkThisText{Text in the same mode as this sentence is #1.}}


%%%%%
%%%%% add in a new TOC style list for requirements
%%%%% http://texblog.org/2008/07/13/define-your-own-list-of/
%%%%%
\usepackage{subfig}
\usepackage[titles,subfigure]{tocloft}% to define requirements list for TOC,F,T
\usepackage{glossaries}
\usepackage{lmodern}
%\usepackage{changebar}
\usepackage[color,DVIps]{changebar}
\cbcolor{blue}


\usepackage{footnote}%https://texblog.org/2012/02/03/using-footnote-in-a-table/


%%%
%%% stuff for bibliographty in multiple parts
%%%
%defernumbers=true,backend=biber,
\usepackage[hyperref=true, backend=biber,natbib=true,style=numeric,sorting=none]{biblatex}

\setcounter{biburlnumpenalty}{100}
\setcounter{biburlucpenalty}{100}
\setcounter{biburllcpenalty}{100}



%%% This makes the pdf file contain active links for all references,
%%% urls, the table of contents, etc., and creates a pdf menu.
\usepackage[pdfborder={0 0 0},plainpages=false,pdfpagelabels,breaklinks,hyperfootnotes=true]{hyperref} %letterpaper depricated (WLC20211009)
\usepackage{url}
%\def\UrlBigBreaks{\do\/\do-\do?}

\usepackage{breakurl}% MUST follow package hyperref

%%% fancy header stuff
\usepackage{fancyhdr}

%%% to fix spacing after punctuation such as with footnotes.[WLC20220719]
%%% it tightens it up but doesn't seem to put footnote marker before the period as it should :(
%%% https://tex.stackexchange.com/questions/492192/unwanted-space-after-footnote-mark
%%% https://tex.stackexchange.com/questions/474362/two-problems-with-fnpct-package
\usepackage{fnpct}


%\usepackage[landscape]{geometry}
\usepackage{pdflscape}



%%% Gets current counter based on chapter, section, subsection, subsubsection, paragraph, subparagraph, ... name
%%% Used to get current reference so when sections are cut from a master document, the \RqtNumberBase can be automatically set
\newcommand{\getcurrentref}[1]{%
  \ifnum\value{#1}=0 ??\else\csname the#1\endcsname\fi
}

% macro for handling multiple requirements from a common set of files
\newcommand{\AUTONUMRQTVAR}{AUTO}
%arg 1 is section level
%arg 2 is manully provided requirement number
\newcommand{\RequirementNumberAM}[2]{\ifthenelse{\equal{\AUTONUMRQTVAR}{AUTO}}{\renewcommand{\RqtNumberBase}{\getcurrentref{#1}}}{\renewcommand{\RqtNumberBase}{#2}}}

%%%%%% old use packages
%%%
%%%\usepackage{xspace}%so we don't have to put braces at end of commands
%%%
%%%\usepackage{svn-multi}
%%%
%%%\usepackage{textcomp}
%%%
%%%\usepackage{fancyhdr}
%%%\usepackage{afterpage}
%%%%\usepackage{ifthen}
%%%
%%%%% This makes the pdf file contain active links for all references,
%%%%% urls, the table of contents, etc., and creates a pdf menu.
%%%\usepackage[pdfborder={0 0 0},plainpages=false,pdfpagelabels]{hyperref} %letterpaper depricated (WLC20211009)
%%%
%%%\usepackage{breakurl}% MUST follow package hyperref
%%%
%%%\usepackage{url}
%%%%\def\UrlBigBreaks{\do\/\do-\do?}
%%%
%%%\usepackage{indentfirst}%WLC20210315 -- added to get 1st lines in sections and below to be indented.
%%%%% This is needed to prevent odd reading when sentences occur after lists and other formatting issues.
%%%
%%%
%%%%%% 20141212WLC - had issues so I edited the text instead
%%%\usepackage{longtable}
%%%\setcounter{LTchunksize}{100}
%%%
%%%\usepackage{csvsimple,longtable,booktabs}
%%%
%%%\usepackage{ifthen}
%%%\usepackage{makeidx}
%%%
%%%\usepackage{titlesec}
%%%
%%%\usepackage{enumerate}
%%%%https://tex.stackexchange.com/questions/119919/no-spacing-between-enumerated-items-with-usepackageenumerate
%%%\usepackage[shortlabels]{enumitem}% to allow set spaces on enumerations, etc.
%%%
%%%
%%%
%%%
%%%%%%
%%%%%% stuff for bibliographty in multiple parts
%%%%%%
%%%%defernumbers=true,backend=biber,
%%%\usepackage[backend=biber,natbib=true,style=numeric,sorting=none]{biblatex}
%%%
%%%\setcounter{biburlnumpenalty}{100}
%%%\setcounter{biburlucpenalty}{100}
%%%\setcounter{biburllcpenalty}{100}
%%%
%%%\usepackage{graphicx}
%%%
%%%\usepackage[nodayofweek]{datetime}
%%%\newdateformat{mydate}{\twodigit{\THEDAY}{ }\shortmonthname[\THEMONTH], \THEYEAR}
%%%\mydate
%%%
%%%\usepackage{float}
%%%\usepackage{subfig}
%%%\usepackage[titles,subfigure]{tocloft}% to define requirements list for TOC,F,T
%%%
%%%\usepackage{pifont}% http://ctan.org/pkg/pifont
%%%\newcommand{\cmark}{\ding{51}}%
%%%\newcommand{\xmark}{\ding{55}}%

%%%%%% icon notes configrues for use in Check Lists and Operation Procedures
% icon notes configrues for use in Check Lists and Operation Procedures
\newcommand{\IconNoteBANG}[1]{\IconNoteBase{MarginIcons/BANG.eps}{#1}}
\newcommand{\IconNoteBULLSEYE}[1]{\IconNoteBase{MarginIcons/BULLSEYE.eps}{#1}}
\newcommand{\IconNoteCLOCK}[1]{\IconNoteBase{MarginIcons/CLOCK.eps}{#1}}
\newcommand{\IconNoteHAND}[1]{\IconNoteBase{MarginIcons/HAND.eps}{#1}}
\newcommand{\IconNoteINFO}[1]{\IconNoteBase{MarginIcons/INFO.eps}{#1}}
\newcommand{\IconNoteKEY}[1]{\IconNoteBase{MarginIcons/KEY.eps}{#1}}
\newcommand{\IconNoteMAGNIFY}[1]{\IconNoteBase{MarginIcons/MAGNIFY.eps}{#1}}
\newcommand{\IconNotePLAYARROW}[1]{\IconNoteBase{MarginIcons/PLAYARROW.eps}{#1}}
\newcommand{\IconNoteRADIATION}[1]{\IconNoteBase{MarginIcons/RADIATION.eps}{#1}}
\newcommand{\IconNoteSHOCK}[1]{\IconNoteBase{MarginIcons/SHOCK.eps}{#1}}
\newcommand{\IconNoteQUESTION}[1]{\IconNoteBase{MarginIcons/QUESTION.eps}{#1}}

%%
%% warning, danger, information, etc. commands for notes with an ICON
%% bsae upon a tabular answer given in https://tex.stackexchange.com/questions/312118/a-note-with-the-dangerous-bend-symbol-how-to-vertically-center-text-and-symbol
\usepackage{array}

\newcommand{\IconNoteBase}[2]{%
    \noindent % I guess this is intended...
    \begin{tabular}{@{}m{0.1\textwidth}@{}m{0.9\textwidth}@{}}%
			&\\\cmidrule{2-2}
			\includegraphics[width=0.6in]{#1} & \bf#2\\\cmidrule{2-2}
    \end{tabular}%
    %\par % ... and this too.
}


%
%%%%%%%%%%%%%%%%%%%%%%%%%%%%%%%%%%%%%%%%%%%%%%%%%%%%%%%%%%%%%%%%%%%%%%%%%%%%%%%%%%%%%%%%%%%%%%%%%%%%%%%%%%%%%%%%%%%%%
% END :: KNEAD \usepackage section
%%%%%%%%%%%%%%%%%%%%%%%%%%%%%%%%%%%%%%%%%%%%%%%%%%%%%%%%%%%%%%%%%%%%%%%%%%%%%%%%%%%%%%%%%%%%%%%%%%%%%%%%%%%%%%%%%%%%%


\newenvironment{my_itemize}{
\begin{itemize}
  \setlength{\itemsep}{1pt}
  \setlength{\parskip}{0pt}
  \setlength{\parsep}{0pt}}{\end{itemize}
}

\newenvironment{my_enumerate}{
\begin{enumerate}
  \setlength{\itemsep}{1pt}
  \setlength{\parskip}{0pt}
  \setlength{\parsep}{0pt}}{\end{enumerate}
}

\newenvironment{tds_enumerate}{
\begin{enumerate}
  \setlength{\itemsep}{1pt}
  \setlength{\parskip}{6pt}
  \setlength{\parsep}{6pt}}{\end{enumerate}
}

\newenvironment{my_description}{
\begin{description}
  \setlength{\itemsep}{1pt}
  \setlength{\parskip}{0pt}
	\setlength{\baselineskip}{0pt}%
  \setlength{\parsep}{0pt}}{\end{description}
}

\newenvironment{traceability_description}%
{\begin{description}{%
  \setlength{\itemsep}{1pt}%
  \setlength{\parskip}{0pt}%
	\setlength{\baselineskip}{0pt}%
  \setlength{\parsep}{0pt}}}%
{\end{description}}


%%%%%%%%%%%%%%%%%%%%%%%%%%%%%%%%%%%%%%%%%%%%%%%%%%%%%%%%%%%%%%%%%%%%%%%%%%%%%%%%%%%%%%%%%%%%%%%%%%%%%%%%%%%%%%%%%%%
% this actually sets the headers and footers, but can be overridden by user before this actually gets instantiated.
%
\lhead{\KNEADleftHeader}
\chead{\KNEADcenterHeader}
\rhead{\KNEADrightHeader}

\lfoot{\KNEADleftFooter}
\cfoot{\KNEADcenterFooter}
\rfoot{\KNEADrightFooter}

% set width of lines below header and above header
\renewcommand{\headrulewidth}{0.4pt}
\renewcommand{\footrulewidth}{0.4pt}
%
%%%%%%%%%%%%%%%%%%%%%%%%%%%%%%%%%%%%%%%%%%%%%%%%%%%%%%%%%%%%%%%%%%%%%%%%%%%%%%%%%%%%%%%%%%%%%%%%%%%%%%%%%%%%%%%%%%%


%%%%%
%%%%% tell it to make the index page(s)
%%%%%
\makeindex


%%%
%%% common command to hide all the details about making the index page print as needed
%%%
\newcommand{\KNEADPrintTheIndexHere}
{
	%%%http://tex.stackexchange.com/questions/23499/incorrect-bookmarks-and-page-number-in-table-of-contents
	\cleardoublepage%%%neeeded to make PDF bookmarks come out correctly
	\phantomsection%%%neeeded to make PDF bookmarks come out correctly
	\addcontentsline{toc}{chapterx}{Index}
	\printindex
}




%%%%%
%%%%% add section formats for paragraph and subparagraph headings
%%%%%
\titleformat{\paragraph}
{\normalfont\normalsize\bfseries}{\theparagraph}{1em}{}
\titlespacing*{\paragraph}
{0pt}{0pt plus 2pt minus 2pt}{0pt plus 2pt minus 2pt}%https://tex.stackexchange.com/questions/56072/vertical-spacing-after-paragraph-heading
%%{0pt}{3.25ex plus 1ex minus .2ex}{1.5ex plus .2ex}%original
%{0pt}{2.5ex plus 1ex minus .2ex}{-3.25ex plus .2ex}% (-) to undo space added by ~// 


%{WLC-20160210}
\titleformat{\subparagraph}
{\normalfont\normalsize\bfseries}{\thesubparagraph}{1em}{}
\titlespacing*{\subparagraph}
{0pt}{0pt plus 2pt minus 2pt}{0pt plus 2pt minus 2pt}%https://tex.stackexchange.com/questions/56072/vertical-spacing-after-paragraph-heading
%{0pt}{-12pt plus 4pt minus 2pt}{-15 pt plus 2pt minus 2pt}%https://tex.stackexchange.com/questions/56072/vertical-spacing-after-paragraph-heading
%%{0pt}{3.25ex plus 1ex minus .2ex}{1.5ex plus .2ex}%original
%{0pt}{2.5ex plus 1ex minus .2ex}{-3.25ex plus .2ex} % (-) to undo space added by ~//

%%%%% command to print DID info at beginning of each section
\newcommand{\DIDINFOBASE}[1]{{
\linespread{0.8}\selectfont  
  {
     \noindent \textsc{#1}  \par
  }
}}
\newcommand{\DIDINFO}[1]{\DIDINFOBASE{#1}}% make sure it's on by default


%%%%%%%%%%%%%%%%%%%%%%%%%%%%%%%%%%%%%%%%%%%%%%%%%%%%%%%%%%%%%%%%%%%%%%%%%%%%%%%%%%%%%%%%%%%%%%%%%%%%%%%%%%%%%%%%%%%%%
% commands to define an \hline that won't allow a page break, ala \\* in longtable
% http://tex.stackexchange.com/questions/6350/how-to-disable-pagebreak-on-hline-in-longtable
%%%%%%%%%%%%%%%%%%%%%%%%%%%%%%%%%%%%%%%%%%%%%%%%%%%%%%%%%%%%%%%%%%%%%%%%%%%%%%%%%%%%%%%%%%%%%%%%%%%%%%%%%%%%%%%%%%%%% 
\makeatletter
\def\nobreakhline{%
\noalign{\ifnum0=`}\fi
 \penalty\@M
\futurelet\@let@token\LT@@nobreakhline}
\def\LT@@nobreakhline{%
\ifx\LT@next\hline
  \global\let\LT@next\@gobble
 \ifx\CT@drsc@\relax
   \gdef\CT@LT@sep{%
     \noalign{\penalty\@M\vskip\doublerulesep}}%
 \else
   \gdef\CT@LT@sep{%
     \multispan\LT@cols{%
       \CT@drsc@\leaders\hrule\@height\doublerulesep\hfill}\cr}%
 \fi
\else
 \global\let\LT@next\empty
 \gdef\CT@LT@sep{%
   \noalign{\penalty\@M\vskip-\arrayrulewidth}}%
\fi
\ifnum0=`{\fi}%
\multispan\LT@cols
 {\CT@arc@\leaders\hrule\@height\arrayrulewidth\hfill}\cr
\CT@LT@sep
\multispan\LT@cols
 {\CT@arc@\leaders\hrule\@height\arrayrulewidth\hfill}\cr
\noalign{\penalty\@M}%
\LT@next}
\makeatother
%%%%%%%%%%%%%%%%%%%%%%%%%%%%%%%%%%%%%%%%%%%%%%%%%%%%%%%%%%%%%%%%%%%%%%%%%%%%%%%%%%%%%%%%%%%%%%%%%%%%%%%%%%%%%%%%%%%%%
% end \nobreakhline
%%%%%%%%%%%%%%%%%%%%%%%%%%%%%%%%%%%%%%%%%%%%%%%%%%%%%%%%%%%%%%%%%%%%%%%%%%%%%%%%%%%%%%%%%%%%%%%%%%%%%%%%%%%%%%%%%%%%%


%%%%%%%%%%%%%%%%%%%%%%%%%%%%%%%%%%%%%%%%%%%%%%%%%%%%%%%%%%%%%%%%%%%%%%%%%%%%%%%
% BEGIN :: Longtable for Chapter 2 Acronyms
%%%%%%%%%%%%%%%%%%%%%%%%%%%%%%%%%%%%%%%%%%%%%%%%%%%%%%%%%%%%%%%%%%%%%%%%%%%%%%%

\newcommand{\KNEADacronymDefinitionArrayStretch}{1.2}
\newlength{\KNEADacronymDefinitionColumnWidth}
\setlength{\KNEADacronymDefinitionColumnWidth}{4.8in}% adjust as necessary based on definitions

\newcommand{\KNEADacronymEntryStart}{\begin{minipage}{\KNEADacronymDefinitionColumnWidth}}
\newcommand{\KNEADacronymEntryClose}{\end{minipage}\\	\hline}
\newcommand{\KNEADacronymEntryCloseLast}{\end{minipage}} % no \\ \hline allowed on final line so last active entry MUST use \KNEADacronymEntryCloseLast


\newcommand{\KNEADChapTwoAcronymTable}[1]
{
\begin{singlespace}
\renewcommand{\arraystretch}{\KNEADacronymDefinitionArrayStretch}%increase as needed for fine tuning line spacing

\begin{center}

\begin{longtable}[H]{|l|l|} %{|@{}l|l@{}|}
\caption{Acronym Definitions}\\
%
\hline \hline
\multicolumn{1}{|l|}{\textbf{Acronym}} & 
\multicolumn{1}{|l|}{\begin{minipage}{\KNEADacronymDefinitionColumnWidth}{\textbf{Definition}}\end{minipage}}\\\hline \hline 
\endfirsthead
%
\hline \hline
\multicolumn{2}{c}%
{{\bfseries Acronym definitions -- continued from previous page}}\\ \hline \hline 
\multicolumn{1}{|l|}{\textbf{Acronym}} & 
\multicolumn{1}{|l|}{\begin{minipage}{\KNEADacronymDefinitionColumnWidth}{\textbf{Definition}}\end{minipage}}\\
\hline \hline 
\endhead
%
\hline \hline 
\multicolumn{2}{|c|}{{\bfseries Acronym definitions continue on next page}}\\ 
\hline \hline
\endfoot
%
\hline\hline 
\multicolumn{2}{|c|}{{\bfseries End of acronym definition table}}\\ 
\hline \hline
\endlastfoot
%
%%% include the project-wide acronyms
\input{#1}%
%
\end{longtable}
\end{center}
\end{singlespace}
}% end \ChapTwoAcronymTable[1]
%%%%%%%%%%%%%%%%%%%%%%%%%%%%%%%%%%%%%%%%%%%%%%%%%%%%%%%%%%%%%%%%%%%%%%%%%%%%%%%
% END :: Longtable for Chapter 2 Acronyms
%%%%%%%%%%%%%%%%%%%%%%%%%%%%%%%%%%%%%%%%%%%%%%%%%%%%%%%%%%%%%%%%%%%%%%%%%%%%%%%

%%%%%%%%%%%%%%%%%%%%%%%%%%%%%%%%%%%%%%%%%%%%%%%%%%%%%%%%%%%%%%%%%%%%%%%%%%%%%%%
% BEGIN :: Longtable for Chapter 2 Glossary
%%%%%%%%%%%%%%%%%%%%%%%%%%%%%%%%%%%%%%%%%%%%%%%%%%%%%%%%%%%%%%%%%%%%%%%%%%%%%%%

\newcommand{\KNEADglossaryTermArrayStretch}{1.0}
\newlength{\KNEADglossaryTermColumnWidth}
\setlength{\KNEADglossaryTermColumnWidth}{4.5in}% adjust as necessary based on definitions
\newlength{\KNEADglossaryBottomSpace}
\setlength{\KNEADglossaryBottomSpace}{6.0pt}

\newcommand{\KNEADglossaryEntryStart}{\begin{minipage}{\KNEADglossaryTermColumnWidth}\vspace{3pt}}
\newcommand{\KNEADglossaryEntryClose}{\vspace{3pt}\end{minipage}\\[\KNEADglossaryBottomSpace]\hline}
\newcommand{\KNEADglossaryEntryCloseLast}{\end{minipage}}% no \\ \hline allowed on final line so last active entry MUST use \KNEADglossaryEntryCloseLast

%
\newcommand{\KNEADChapTwoGlossaryTable}[1]
{
\begin{singlespace}

\begin{center}

\renewcommand*{\arraystretch}{\KNEADglossaryTermArrayStretch}%increase as needed for fine tuning line spacing
\begin{longtable}[H]{|l|l|} %{|@{}l|l@{}|}
\caption{Glossary Terms and Definitions} \\
%
\hline \hline
\multicolumn{1}{|l|}{\textbf{Glossary Term}} & 
\multicolumn{1}{|l|}{\begin{minipage}{\KNEADglossaryTermColumnWidth}{\textbf{Definition}}\end{minipage} } \\
\hline \hline 
\endfirsthead
%
\multicolumn{2}{c}%
{{\bfseries Glossary terms -- continued from previous page}}\\ \hline \hline 
\multicolumn{1}{|l|}{\textbf{Glossary Term}} & 
\multicolumn{1}{|l|}{\begin{minipage}{\KNEADglossaryTermColumnWidth}{\textbf{Definition}}\end{minipage}}\\\hline \hline 
\endhead
%
\hline \hline \multicolumn{2}{|c|}{{\bfseries Glossary terms continue on next page}}\\ \hline \hline
\endfoot
%
\hline\hline \multicolumn{2}{|c|}{{\bfseries End of glossary terms table}}\\ \hline \hline
\endlastfoot
%
%%% include the project-wide acronyms
\input{#1}
%
\end{longtable}
\end{center}
\end{singlespace}
}% end \ChapTwoGlossaryTable[1]
%%%%%%%%%%%%%%%%%%%%%%%%%%%%%%%%%%%%%%%%%%%%%%%%%%%%%%%%%%%%%%%%%%%%%%%%%%%%%%%
% End :: Longtable for Chapter 2 Glossary
%%%%%%%%%%%%%%%%%%%%%%%%%%%%%%%%%%%%%%%%%%%%%%%%%%%%%%%%%%%%%%%%%%%%%%%%%%%%%%%

%%%%%%%%%%%%%%%%%%%%%%%%%%%%%%%%%%%%%%%%%%%%%%%%%%%%%%%%%%%%%%%%%%%%%%%%%%%%%%%
% BEGIN :: Standard Bibliography command for Chapter 2 reference citations
%%%%%%%%%%%%%%%%%%%%%%%%%%%%%%%%%%%%%%%%%%%%%%%%%%%%%%%%%%%%%%%%%%%%%%%%%%%%%%%
% standard macro to include all three levels of references in Chapter 2.
% redefine this at Project or Artifact level as needed to change levels, etc.
\newcommand{\KNEADChapTwoReferences}
{
This section lists the referenced documents for this document.
The references are categorized into two categories:

\begin{description}
	\item[External] Documents not directly associated with this project.
	\item[Project] Documents that are directly associated with this project.
	%\item[Artifact] Documents that are directly associated with just this artifact. This maybe would include items such as separated appendices.
\end{description}

\subsection{External Documents}
\label{loc__Refs_ExternalDocuments}

\printbibliography[heading=none,keyword=KNEAD_BiberGlobal]


\subsection{Project Specific Documents}
\label{loc__Refs_ProjectDocuments}

\printbibliography[heading=none,keyword=KNEAD_BiberProject]


%\subsection{Artifact Specific Documents}
%\label{loc:Refs_ArtifactDocuments}
%
%\printbibliography[heading=none,keyword=KNEAD_BiberArtifact]
}
%
%%%%%%%%%%%%%%%%%%%%%%%%%%%%%%%%%%%%%%%%%%%%%%%%%%%%%%%%%%%%%%%%%%%%%%%%%%%%%%%
% END :: Standard Bibliography command for Chapter 2 reference citations
%%%%%%%%%%%%%%%%%%%%%%%%%%%%%%%%%%%%%%%%%%%%%%%%%%%%%%%%%%%%%%%%%%%%%%%%%%%%%%%


%%%%% Manually numbered requirement
%%%%% table for requirements

\newboolean{boKNEADshowSpecNotes}
\setboolean{boKNEADshowSpecNotes}{true}

\newcommand{\CBVERSION}{UNDEFINED}

\newcommand{\ChangedText}[2]{\ifthenelse{\equal{\CBVERSION}{#1}}{\cbstart #2 \cbend}{#2}}

\newcounter{NRQMT}
\newlistof[NRQMT]{rqtspecification}{aux_spec}{\listspecificationsname}

\newcommand{\numberedrqtspecification}[2]{%
\refstepcounter{rqtspecification}
\par\noindent\textbf{Specification {#1} {#2}}
\addcontentsline{aux_spec}{rqtspecification}{\protect\numberline{#1}#2}\par}

\newcommand{\OneRqmtThreshold}{({\tt T}) - Threshold}
\newcommand{\OneRqmtObjective}{({\tt O}) - Objective}
\newcommand{\OneRqmtInactive} {({\tt I}) - Inactive}
\newcommand{\OneRqmtDeleted}  {({\tt D}) - Deleted}


\newlength{\RqmtMiniPageWidth}
\if@dobindprt % set margins to 1.5" on left (leave room for binding) and 1.0" right
\setlength{\RqmtMiniPageWidth}{4.7in}
\else%!dobindprt so set margins to 1.0" on left and right sides
\setlength{\RqmtMiniPageWidth}{5.2in}
\fi%dobindprt


%%%%
%%%%
%%%% Simple table for listing single requirements with all applicable stuff
%%%%
%%%% #1 is requirement number
%%%% #2 is requirement name
%%%% #3 is requirement label (expected to be of form rqt:XXX)
%%%% #4 is text of the single requirement

%%%% #5 is version that defines this item
%%%% #6 is status (must be a list of \items of form [phase] active, inactive, etc.},#7 is version that defines this item
%%%% #8 is acceptance, #9 is version that defines this item
%%%% #10 is traceability, #11 is version that defines this item
%%%% #12 is notes {must be a list of \items}, #13 is version that defines this item
%%%%
%%%%

\newcommand{\ONERQMTV}[9]{
\renewcommand{\therqtspecification}{#1}
\begin{table}[H]%\usepackage{booktabs} for \toprule, \midrule, \bottomrule
\ifthenelse{\equal{\CBVERSION}{#9}}{\cbstart}{}
\begin{tabular}{@{}r|l@{}}
\toprule
\toprule            
%SPECIFICATION name for list of specifications    
    \multicolumn{2}{@{}c@{}}{%this c only centers the minipage
	   \begin{minipage}{\textwidth}% to allow for it to wrap
	    \centering% to make contents of minipage to be centered
	     \numberedrqtspecification{#1}{#2}
	      \label{#3}
	   \end{minipage}} \\
\midrule            
% TEXT is the formal requirement specification text
			{\bf Text}  & 
			\begin{minipage}{\RqmtMiniPageWidth}
			   {#4} % e.g. The system interface 1 connector 1 as style 38999-XYZ labeled as J1.  
			\end{minipage} \\
\midrule
% STATUS, these are the requirement status by phases
			{\bf Status} &
			\begin{minipage}{\RqmtMiniPageWidth} %%% minipage to allow for multiple specifics in one cell
			 \begin{description}[topsep=0pt,itemsep=-1ex,partopsep=1ex,parsep=1ex]
			    {#5} % MUST BE a list of decription items -- of form \item [Phase] status				 
			 \end{description}  
			\end{minipage} \\			
\midrule
% ACCEPTANCE method (inspection, demonstration, test, analysis, or other		 
			{\bf Acceptance}   & 
			\begin{minipage}{\RqmtMiniPageWidth} %%% minipage to allow for multiple lines of verification			
			   {#6} %e.g. This specification shall be verified by inspection. 
			        % could include other test info here but the formatting is left to user to provide
			\end{minipage} \\
\midrule 
% TRACEABILITY provides a reference to a higher level specification document, if applicable		
			{\bf Traceability} & 
			\begin{minipage}{\RqmtMiniPageWidth} %%% minipage to allow for multiple lines of traceability
			 \begin{description}[topsep=0pt,itemsep=-1ex,partopsep=1ex,parsep=1ex]
			    {#7} % MUST BE a list of decription items -- of form \item [Document] requirement number				 
			 \end{description}						
			%\end{minipage}\\				
%\midrule 
%print out the notes if we're supposed to
% NOTES	are NOT part of the specification, just a place for comments about the specification
\ifthenelse{\boolean{boKNEADshowSpecNotes}}{%	
\end{minipage}\\% end Traceability minipage in the ifthenelse to keep \midrule after the break			
\midrule% seems to need to be right after the break, thus the placement of end{minipage}\\ in the ifthenelse					
			{\bf Notes}        & 
			\begin{minipage}{\RqmtMiniPageWidth} %%% minipage to allow for multiple lines of notes
			 \begin{enumerate}[topsep=0pt,itemsep=-1ex,partopsep=1ex,parsep=1ex]
			    {#8} % MUST BE a list of enumerated items -- of form \item Note 1...
				       % even if there is just one, e.g. \item No notes for this specification.		 
			 \end{enumerate}			    
			\end{minipage}\\	
\bottomrule% end table with double bottomrule right after the final end{minipage} and break
\bottomrule
}%end ifthen clause
{%else just end the table by ending traceability minipage, break, and double bottomrules
\end{minipage}\\				
\bottomrule 
\bottomrule}%end ifthenelse		
\end{tabular}
\ifthenelse{\equal{\CBVERSION}{#9}}{\index{All Changes This Version}\cbend}{}							
\end{table}
} %end of ONERQMTV[9]


%https://tex.stackexchange.com/questions/2132/how-to-define-a-command-that-takes-more-than-9-arguments
\newcommand{\MULTIRQMTV}[1]%looks like [10], #1 is saved and #2...#10 --> #1...#9 in MULTIRQMTVbase[9]
{
\def\SpecNumber{#1}
\MULTIRQMTVbase
}

\newcommand{\MULTIRQMTVbase}[9]{
\renewcommand{\therqtspecification}{\SpecNumber}
\begin{table}[H]%\usepackage{booktabs} for \toprule, \midrule, \bottomrule
\ifthenelse{\equal{\CBVERSION}{#9}}{\cbstart}{}
\begin{tabular}{@{}r|l@{}}
\toprule
\toprule            
%SPECIFICATION name for list of specifications    
    \multicolumn{2}{@{}c@{}}{%this c only centers the minipage
	   \begin{minipage}{\textwidth}% to allow for it to wrap
	    \centering% to make contents of minipage to be centered
	     \numberedrqtspecification{\SpecNumber}{#1}
	      \label{#2}
	   \end{minipage}} \\
\midrule            
% SYNOPSIS is just a quick summary, not the formal rqt specification
			{\bf Text}  & 
			\begin{minipage}{\RqmtMiniPageWidth}
				{#3} %The system interface 1 connector 1 as style 38999-XYZ labeled as J1.  
			\end{minipage} \\
\midrule
% SPECIFICS, these are the formal specification details
			{\bf Specifics} &
			\begin{minipage}{\RqmtMiniPageWidth} %%% minipage to allow for multiple specifics in one cell
			 \begin{enumerate}[topsep=0pt,itemsep=-1ex,partopsep=1ex,parsep=1ex]
				{#4} 
%%%%%			 		\item The connector for interface 1 shall be labled as J1. 
%%%%%				    \item The connector for interface 1 shall be MFG P/N 38999-XYZ.  
%%%%%					\item The connector for interface 1 shall have finish type \TBD.
%%%%%					\item The connector for interface 1 shall have finish color \TBD.
%%%%%					\item The connector for interface 1 shall be keyed as \TBD.							 
			 \end{enumerate}  
			\end{minipage} \\
\midrule
%STATUS  			 
			{\bf Status}   & 
			\begin{minipage}{\RqmtMiniPageWidth} %%% minipage to allow for multiple lines of verification	
				\begin{description}[topsep=0pt,itemsep=-1ex,partopsep=1ex,parsep=1ex]
			   {#5} % MUST BE a list of decription items -- of form \item [Phase] status
				\end{description}
			\end{minipage} \\
\midrule
% ACCEPTANCE method (inspection, demonstration, test, analysis, or other
			{\bf Acceptance}       & 
			\begin{minipage}{\RqmtMiniPageWidth} %%% minipage to allow for multiple lines of status			
			   {#6} %Active 
			\end{minipage} \\	
\midrule
% TRACEABILITY provides a reference to a higher level specification document, if applicable		
			{\bf Traceability} & 
			\begin{minipage}{\RqmtMiniPageWidth} %%% minipage to allow for multiple lines of traceability	
					\begin{description}[topsep=0pt,itemsep=-1ex,partopsep=1ex,parsep=1ex]
			   {#7} %This specification is derived from \ldots.
					\end{description}
			%\end{minipage}\\
%\midrule
%print out the notes if we're supposed to
% NOTES	are NOT part of the specification, just a place for comments about the specification
\ifthenelse{\boolean{boKNEADshowSpecNotes}}{%
\end{minipage}\\% end Traceability minipage in the ifthenelse to keep \midrule after the break			
\midrule% seems to need to be right after the break, thus the placement of end{minipage}\\ in the ifthenelse						
			{\bf Notes}        & 
			\begin{minipage}{\RqmtMiniPageWidth} %%% minipage to allow for multiple lines of notes		
				\begin{enumerate}[topsep=0pt,itemsep=-1ex,partopsep=1ex,parsep=1ex]
				{#8} % MUST BE a list of enumerated items -- of form \item Note 1...
				     % even if there is just one, e.g. \item No notes for this specification.		 
				\end{enumerate}
			\end{minipage} \\	
\bottomrule% end table with double bottomrule right after the final end{minipage} and break
\bottomrule
}%end ifthen clause
{%else just end the table by ending traceability minipage, break, and double bottomrules
\end{minipage}\\				
\bottomrule 
\bottomrule}%end ifthenelse			
\end{tabular}
\ifthenelse{\equal{\CBVERSION}{#9}}{\index{All Changes This Version}\cbend}{}							
\end{table}
}%end of MULTIRQMTVbase[9]




%aliases to find KPP and KSA requirements for the B-spec
\newcommand{\ONERQMTVKPP}{\ONERQMTV}
\newcommand{\ONERQMTVKSA}{\ONERQMTV}
\newcommand{\MULTIRQMTVKPP}{\MULTIRQMTV}
\newcommand{\MULTIRQMTVKSA}{\MULTIRQMTV}
% old, deprecated ones
\newcommand{\ONERQMTKPP}{\ONERQMT}
\newcommand{\ONERQMTKSA}{\ONERQMT}

%%%%%%%%%%%%%%%%%%%%%%%%%%%%%%%%%%%%%%%%%%%%%%%%%%%%%%%%%%%%%%%%%%%%%%%%%%%%%%%%%%%%%%%%%%%%%%%%%%%
%%%%%%%%%%%%%%%%%%%%%%%%%%%%%%%%%%%%%%%%%%%%%%%%%%%%%%%%%%%%%%%%%%%%%%%%%%%%%%%%%%%%%%%%%%%%%%%%%%%
%% Deprecated stuff below here, shouldn't be used, migrate to ones above
%%%%%%%%%%%%%%%%%%%%%%%%%%%%%%%%%%%%%%%%%%%%%%%%%%%%%%%%%%%%%%%%%%%%%%%%%%%%%%%%%%%%%%%%%%%%%%%%%%%
%%%%%%%%%%%%%%%%%%%%%%%%%%%%%%%%%%%%%%%%%%%%%%%%%%%%%%%%%%%%%%%%%%%%%%%%%%%%%%%%%%%%%%%%%%%%%%%%%%%


%%\newcounter{NRQMT}
%%\newlistof[NRQMT]{rqtspecification}{aux_spec}{\listspecificationsname}
%%%%
%%%%
%%%% Simple table for listing single requirements with all applicable stuff
%%%%
%%%% #1 is requirement number
%%%% #2 is requirement name
%%%% #3 is requirement label (expected to be of form rqt:XXX)
%%%% #4 is text of the single requirement
%%%% #5 is status (must be a list of \items of form [phase] active, inactive, etc.}
%%%% #6 is acceptance
%%%% #7 is traceability
%%%% #8 is notes {must be a list of \items}
%%%%
%%%%
\newcommand{\ONERQMT}[8]{
\renewcommand{\therqtspecification}{#1}
\begin{table}[H]%\usepackage{booktabs} for \toprule, \midrule, \bottomrule
\begin{tabular}{@{}r|l@{}}
\toprule
\toprule            
%SPECIFICATION name for list of specifications    
    \multicolumn{2}{@{}c@{}}{%this c only centers the minipage
	   \begin{minipage}{\textwidth}% to allow for it to wrap
	    \centering% to make contents of minipage to be centered
	     \numberedrqtspecification{#1}{#2}
	      \label{#3}
	   \end{minipage}} \\
\midrule            
% TEXT is the formal requirement specification text
			{\bf Text}  & 
			\begin{minipage}{\RqmtMiniPageWidth}
			   {#4} % e.g. The system interface 1 connector 1 as style 38999-XYZ labeled as J1.  
			\end{minipage} \\
\midrule
% STATUS, these are the requirement status by phases
			{\bf Status} &
			\begin{minipage}{\RqmtMiniPageWidth} %%% minipage to allow for multiple specifics in one cell
			 \begin{my_description}
			    {#5} % MUST BE a list of decription items -- of form \item [Phase] status				 
			 \end{my_description}  
			\end{minipage} \\			
\midrule
% ACCEPTANCE method (inspection, demonstration, test, analysis, or other		 
			{\bf Acceptance}   & 
			\begin{minipage}{\RqmtMiniPageWidth} %%% minipage to allow for multiple lines of verification			
			   {#6} %e.g. This specification shall be verified by inspection. 
			        % could include other test info here but the formatting is left to user to provide
			\end{minipage} \\
\midrule 
% TRACEABILITY provides a reference to a higher level specification document, if applicable		
			{\bf Traceability} & 
			\begin{minipage}{\RqmtMiniPageWidth} %%% minipage to allow for multiple lines of traceability
			 \begin{my_description}
			    {#7} % MUST BE a list of decription items -- of form \item [Document] requirement number				 
			 \end{my_description}						
			\end{minipage}\\				
\if@showreqnotes
\midrule 
% NOTES	are NOT part of the specification, just a place for comments about the specification
			{\bf Notes}        & 
			\begin{minipage}{\RqmtMiniPageWidth} %%% minipage to allow for multiple lines of notes
			 \begin{my_enumerate}
			    {#8} % MUST BE a list of items
				     % even if there is just one -- of form \item blah blah			 
			 \end{my_enumerate}			    
			\end{minipage} \\	
\fi%showreqnotes
\bottomrule 								
\bottomrule 								
\end{tabular}
\end{table}
} %end of ONERQMT[8]




















\newcommand{\NRQMTT}[9]{
\renewcommand{\therqtspecification}{#1}
\begin{table}[H]%\usepackage{booktabs} for \toprule, \midrule, \bottomrule


% longtable stuff, still trying to get it to work 20141212WLC
%\begin{longtable}[H]%\usepackage{booktabs} for \toprule, \midrule, \bottomrule
%\hline \hline
%\endhead
%\hline \hline
%\endfirsthead
%\hline \multicolumn{1}{|r|}{{Continued on next page}} \\ \hline
%\endfoot
%\hline \hline
%\endlastfoot

\begin{tabular}{@{}r|l@{}}
\toprule
\toprule            
%SPECIFICATION name for list of specifications    
    \multicolumn{2}{@{}c@{}}{%this c only centers the minipage
	   \begin{minipage}{\textwidth}% to allow for it to wrap
	    \centering% to make contents of minipage to be centered
	     \numberedrqtspecification{#1}{#2}
	      \label{#3}
	   \end{minipage}} \\
\midrule            
% SYNOPSIS is just a quick summary, not the formal rqt specification
			{\bf Text}  & 
			\begin{minipage}{\RqmtMiniPageWidth}
				{#4} %The system interface 1 connector 1 as style 38999-XYZ labeled as J1.  
			\end{minipage} \\
\midrule
% SPECIFICS, these are the formal specification details
			{\bf Specifics} &
			\begin{minipage}{\RqmtMiniPageWidth} %%% minipage to allow for multiple specifics in one cell
			 \begin{my_enumerate}
				{#5} 
%%%%%			 		\item The connector for interface 1 shall be labled as J1. 
%%%%%				    \item The connector for interface 1 shall be MFG P/N 38999-XYZ.  
%%%%%					\item The connector for interface 1 shall have finish type \TBD.
%%%%%					\item The connector for interface 1 shall have finish color \TBD.
%%%%%					\item The connector for interface 1 shall be keyed as \TBD.							 
			 \end{my_enumerate}  
			\end{minipage} \\			
\midrule
% ACCEPTANCE method (inspection, demonstration, test, analysis, or other
			{\bf Acceptance}       & 
			\begin{minipage}{\RqmtMiniPageWidth} %%% minipage to allow for multiple lines of status			
			   {#6} %Active 
			\end{minipage} \\	
\midrule
%%% with the switch to marking each line of the specification, we no longer need to show the overall status
%%% but we leave this option for older documents, just need to add this to options until \RQMTxxx calls are fixed
%STATUS  			 
			{\bf Status}   & 
			\begin{minipage}{\RqmtMiniPageWidth} %%% minipage to allow for multiple lines of verification	
				\begin{my_description}
			   {#7} % MUST BE a list of decription items -- of form \item [Phase] status
				\end{my_description}
			\end{minipage} \\
\midrule 
% TRACEABILITY provides a reference to a higher level specification document, if applicable		
			{\bf Traceability} & 
			\begin{minipage}{\RqmtMiniPageWidth} %%% minipage to allow for multiple lines of traceability	
					\begin{my_description}
			   {#8} %This specification is derived from \ldots.
					\end{my_description}
			\end{minipage}\\				
\if@showreqnotes
\midrule 
% NOTES	are NOT part of the specification, just a place for comments about the specification
	
			{\bf Notes}        & 
			\begin{minipage}{\RqmtMiniPageWidth} %%% minipage to allow for multiple lines of notes		
				\begin{my_enumerate}
				{#9} %There are no notes for this specification.
				\end{my_enumerate}
			\end{minipage} \\	
\fi%showreqnotes
\bottomrule 								
\bottomrule 								
\end{tabular}
\end{table}
%\end{longtable}
}%end of NRQMT[9]


